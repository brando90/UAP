\documentclass[a4paper]{article}

\usepackage[english]{babel}
\usepackage[utf8]{inputenc}
\usepackage{amsmath}
\usepackage{graphicx}
\usepackage[colorinlistoftodos]{todonotes}

\title{EstimatorDB}

\author{Pedro Cattori and Brando Miranda}

\begin{document}
\maketitle

\begin{abstract}
In the modern world of big data, some times it is too costly to go through all the data one may have in a database.
However, we are still interested in being able to query a database holding our data and get some understanding of the data that we have.
In this paper I propose two different methods to estimate the total amount of a quantity some group of you data may have.
First we will approximate the number of elements pertaining to one group and then, we will also estimate their mean value.
With these two approximated quantities, we can easily estimate the total amount one group contributes by multiplying both averages.
I will also prove the correctness of the algorithm that I propose.
\end{abstract}



\end{document}